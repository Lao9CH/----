\documentclass{report}

\usepackage[UTF8]{ctex}

\input{preamble}
\input{macros}
\input{letterfonts}

\title{\Huge{高中数学\\练习册}}
\author{\huge{LaoCH数学}}
\date{}

\begin{document}

\maketitle

\section{集合:元素的认识}

\dfn{元素的认识}{\
    设$ A=\{ 1,2,3,......,100\} ,B=\{x+y|x,y \in A\} $ ,问A 有几个元素?
    }
\pf{分析}{本题考查了集合的元素的理解。认识集合,理解元素的含义很重要!}
\vskip 30ex
\qs{元素的认识}{
    设$ A=\{ $直线$ \} ,B=\{ $抛物线$ \} $ ,问$ A\cap B $ 有几个元素?
}
\vskip 30ex

\nt{元素的认识\\::\\::}{    }

\newpage


\section{数学练习:第二天}
\dfn{元素的认识}{  
    
}

\pf{分析}{   }

\vskip 30ex

\qs{典型练习}{
    
}
\vskip 30ex

\nt{::\\::}{    }

\newpage



\section{概率:三人比赛问题}
\dfn{三人比赛}{
      小明与甲, 乙二位同学进行一场乒乓球比赛, 每局两人比赛, 没有平局, 一局决出 胜负. 已知每局比赛小明胜甲的概率为 
     $\frac{1}{4}$ , 小明胜乙的概率为  $\frac{2}{5} $, 甲胜乙的概率为  $\frac{2}{3}$ , 比赛胜负 间互不影响. 
     规定先由其中 2 人进行第一局比赛, 后每局胜者再与此局未比赛的人进行下一局 的比赛, 在比赛中某人首先获胜两局就成为这次
     比赛的获胜者, 比赛结束. 因为小明是三人 中水平最弱的, 所以让小明决定第一局的两个比赛者(小明可以选定自己比赛, 
     也可以选定甲、 乙比赛).\\ (I )若小明选定第一局由甲、乙比赛, 求 “只进行三局, 小明就成为获胜者” 的概率;\\
     (II)请帮助小明进行第一局的决策, 使得小明最终成为获胜者的概率最大, 说明理由.
}

\pf{分析}{ 运用工具:树状图 }
\vskip 30ex
\qs{三人比赛}{
    为了丰富业余生活,甲、乙、丙三人进行羽毛球比赛.比赛规则如下:(1)每场比赛有两人 参加,
    并决出胜负;(2)每场比赛获胜的人与未参加此场比赛的人进行下一场的比赛;
    (3) 依次循环,直到有一个人首先获得两场胜利,则本次比赛结束,此人为本次比赛的冠军. 
    已知在每场比赛中,甲胜乙的概率为 $\frac{2}{3}$ ,甲胜丙的概率为 $\frac{3}{5}$ ,
    乙胜丙的概率为 $\frac{1}{2}$.\\
(1)求甲和乙先赛且共进行 4 场比赛的概率;\\
(2)请通过计算说明,哪两个人进行首场比赛时,甲获得冠军的概率最大?
}

\pf{解答}{}
\vskip 30ex

\nt{::\\::}{    }

\newpage


\section{数学练习:第二天}
\dfn{典型例题}{   
    
}

\pf{分析}{   }

\vskip 30ex

\qs{典型练习}{
    
}
\vskip 30ex

\qs{典型练习}{
    
}
\vskip 30ex

\nt{::\\::}{    }

\newpage


\end{document}
